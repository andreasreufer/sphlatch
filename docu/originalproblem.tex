%% original problem

\section{A problem and one of its solutions}

\subsection{the N-Body problem}

$O(N^2)$ of N-Body calculation

short-range, long-range physics

\subsection{the Barnes \& Hut tree as a solution}
%do not mention implementation stuff

$O( N \log N )$\\
- hierarchical representation

As shown in \cite{1993ApJ...414..200M} the multipole moments relative to the center of mass 
\begin{equation}
X_{i} = \frac{1}{M} \sum_{c} m_{c} x_{i, c} ~~~~~~~~ \text{for}~M > 0
\end{equation}

of a clump of particles with masses $m_{c}$ and positions $\xvec_{c}$ are given by the following equations for the rank-0 monopole tensor (or the common mass)
\begin{equation}
M = \sum_{c} m_c \\
\end{equation}

the symmetric and traceless rank-2 quadrupole tensor 
\begin{equation}
Q_{ij} = \sum_{c} m_c \big( 3 r_{i, c} r_{j, c} - \rvec_{c}^2 \delta_{i j} \big) \\
\end{equation}

(where $\rvec = \xvec_{c} - \Xvec $ denotes the position of the particle relative to the center of mass), the rank-3 octupole tensor which can be reduced to a rank-2 tensor
\begin{equation}
S_{ij}  = \sum_{k} m_c \big[ 5 \big( 3 - 2  \delta_{i j} \big) r_{i, c}^2 - 3 \rvec_{k}^2 \big] r_{j, c}\\
\end{equation}

and a rank-0 tensor component
\begin{equation}
S_{1 2 3} = 15 \sum_{k} m_c r_{1, c} r_{2, c} r_{3, c}
\end{equation}

This representation of multiple particles as a clump with a center of mass and its multipole moments is a special case of the more general case, where the constituents of the parent clumps are also clumps with multipole moments. The generalization gives the following equations where $c$ indexes the children clumps. For the parent clump the monopole moment yields

\begin{equation}
\label{eq_monopole}
M = \sum_{c} M_{c}
\end{equation}

the quadrupole tensor
\begin{equation}
\label{eq_quadrupole}
Q_{ij} = \sum_{c} M_{c} \big( 3 r_{i, c} r_{j, c} - \rvec_{c}^{2} \delta_{i j} \big) + Q_{i j, c} 
\end{equation}

the octupole tensor components
\begin{align}
\label{eq_octupole}
S_{ij} =& \sum_{c}  M_{c} \big[ 5 \big( 3 - 2 \delta_{i,j} \big) r_{i, c}^{2} - 3 \rvec_{c}^2 \big] r_{j, c} 
+ 5 \big( 1 - \delta_{i j} \big) r_{i, c} Q_{i j, c} \nonumber \\
&+ \frac{5}{2} r_{j, c} Q_{i j, c} - \sum_{l} \big[ r_{l, c} Q_{j l, c} \big] + S_{i j, c} \\
S_{1 2 3} =& 15 \sum_{c} M_{c} r_{1, c} r_{2, c} r_{3, c} + \frac{5}{3} \big( r_{1, c} Q_{2 3, c} + r_{2, c} Q_{3 1, c} + r_{3, c} Q_{1 2, c} \big) + S_{1 2 3, c }
\end{align}

and finally the center of mass yields
\begin{equation}
X_{i} = \frac{1}{M} \sum_{c} M_{c} X_{i, c} ~~~~~~~~  M > 0
\end{equation}

So particles can be treated like clumps with only a monopole moment and vanishing higher order multipoles.\\

The gravitational potential of a clump can now be approximated in terms of its multipole moments

\begin{equation}
\phi(\rvec) = G \bigg(
\underbrace{ - \frac{M}{r} }_{monopole} ~ 
\underbrace{ - \frac{Q_{i j}}{r^{3}} \frac{r_{i} r_{j} }{2 r^{2} } }_{quadrupole}~ 
\underbrace{ - \frac{S_{i j}}{r^{4}} \frac{r_{i}^2 r_{j}}{2 r^{3} } + \frac{ S_{1 2 3} }{r^{4}}\frac{ r_{1} r_{2} r_{3} }{2 r^{3}} }_{octupole}~ 
+ O\Big(\frac{1}{r^{7}}\Big) \bigg)
\end{equation}

and so we get the resulting acceleration for a point mass at $\rvec$ in this potential
\begin{align}
a_{k} &= - \nabla_{k} \phi(\rvec) \approx - G \bigg(
\underbrace{ \frac{M}{r^{2}} \frac{r_{k}}{r} }_{monopole}+ 
\underbrace{ \frac{Q_{i j}}{r^{4}}
\big( \frac{\delta_{i k} r_{j} }{2r} + \frac{5 r_{i} r_{j} r_{k} }{2r^{3}} \big)
}_{quadrupole} \nonumber \\
&+ \underbrace{ \frac{S_{i j}}{r^{5}}
\big( \frac{ \delta_{i k} r_{i} r_{j} }{r^{2}}
+ \frac{ \delta_{j k} r_{i}^{2} }{2 r^{2} }
- \frac{7 r_{i}^{2} r_{j} r_{k} }{r^{4} } \big) 
 + \frac{S_{1 2 3}}{r^{5} }
\big( \frac{\delta_{1 k} r_{2} r_{3} + \delta_{2 k} r_{3} r_{1} + \delta_{3 k} r_{1} r_{2}}{2 r^{2}} 
- \frac{7 r_{1} r_{2} r_{3} r_{k} }{2 r^{4}}\big) 
}_{octupole}
\bigg)
\end{align}

$\theta$ clumping parameter, self-acceleration for $\theta \ge 1$
\cite{1986Natur.324..446B}